\documentclass[11pt,a4paper]{article}
\usepackage{a4wide}
\usepackage[utf8]{inputenc}
\usepackage[T2A]{fontenc}
\usepackage{graphics,graphicx,epsfig}
\usepackage{amssymb,amsfonts,amsthm,amsmath,mathtext,cite,enumerate,float}
\usepackage[english,russian]{babel}
\usepackage[all]{xy}
\usepackage{morefloats}
\usepackage{pgf}
\usepackage[debug,outputdir={docgraphs/}]{dot2texi}
\usepackage{tikz}
\usepackage{scalefnt}
\usepackage{float}
\usepackage{verbatim}
\usepackage{placeins}
\usepackage{url}
\usepackage{babelbib}

\makeatletter
\def\@settitle{\begin{center}%
    \baselineskip14\p@\relax
    \bfseries
    \@title
  \end{center}%
}

\makeatother

\newcommand{\bomega}{\boldsymbol{\omega}}

\begin{document}

\begin{center}
  МОДИФИКАЦИЯ АЛГОРИТМА ЛЕВЕНБЕРГА-МАРКВАРДТА ДЛЯ ЗАДАЧ НЕЛИНЕЙНОЙ РЕГРЕССИИ С УЧЕТОМ
  ПОГРЕШНОСТЕЙ КАК В ЗАВИСИМЫХ, ТАК И В НЕЗАВИСИМЫХ ДАННЫХ

  \bigskip
  Г.\,И.~Рудой
\end{center}

\section{Введение}

Для известной задачи нахождения оптимальных коэффициентов некоторой фиксированной
регрессионной модели, представленной в виде формулы, по набору экспериментальных
данных широко применяется алгоритм
Левенберга-Марквардта \cite{Marquardt1963Algorithm}. Однако, данный алгоритм построен
и статистически обоснован в предположении о нормальности распределения регрессионных
остатков и точно измеренных независимых переменных~--- иными словами, учитываются
и рассматриваются только ошибки измерения зависимой переменной. Более того,
предполагается, что ошибки для всех точек принадлежат одному и тому же распределению
с одними и теми же параметрами.

В ряде физических приложений это предположение не выполняется. Например, в задаче
нахождения дисперсионной зависимости прозрачного полимера (то есть, зависимости
коэффициента преломления $n$ от длины волны $\lambda$) погрешности измерения
различных физических параметров, вообще говоря, различны. Так, например, если
для измерения длины волны $\lambda$ используется дифракционная решетка, то постоянной
является относительная погрешность определения длины волны
$\frac{\sigma_{\lambda_i}}{\lambda_i} \approx \text{const}$, и, следовательно,
погрешность определения длины волны зависит от самой длины волны.

Таким образом, возникает задача поиска оптимальных коэффициентов регрессионной
формулы с учетом отличающихся погрешностей различных экспериментальных точек.
Для некоторых частных случаев эта задача уже была решена: например, 
в работе \cite{...} вводится предположение, что зависимые переменные $y_i$ измеряются
неточно, и каждая переменная $y_i$ имеет свою собственную погрешность измерения
$\sigma_{y_i}$. Затем в работе показывается, что обычный функционал суммы квадратов
регрессионных остатков, где каждый остаток нормирован на соответствующую величину
$\sigma_{y_i}^2$, корректен и статистически состоятелен.

В настоящей работе вводятся дополнительные предположения о том, что независимые
переменные также измеряются неточно, и каждая переменная имеет свою собственную
погрешность измерения. Предлагается функционал качества и модифцированный алгоритм
Левенберга-Марквардта, позволяющий найти оптимальные параметры согласно этому
функционалу качества и опирающийся на классический алгоритм
Левенберга-Марквардта (в дальнейшем будем называть их мАЛМ и АЛМ соответственно).
Доказывается сходимость модифицированного алгоритма и приводятся
результаты на экспериментальных данных по измерению насыщения лазерного излучателя.

\section{Постановка задачи}

Дана обучающая выборка $D = \{ \mathbf{x}_i, y_i \} | i \in \{ 1, \dots, \ell \}, x_i \in \mathbb{R}^m, y_i \in \mathbb{R}$.
Для каждой зависимой переменной переменной $y_i$ известно
стандартное отклонение ошибки ее измерения $\sigma_{y_i}$, а для соответствующего
вектора независимых переменных $\mathbf{x}_i$ аналогично известны стандартные
отклонения его компонент $\sigma_{x_{ij}} | j \in \{ 1, \dots, m \}$.
Пусть, кроме того, дана некоторая параметрическая регрессионная модель
$y = f (\mathbf{x}, \bomega)$.

Для удобства обозначим вектор, составленный из ошибок измерений зависимых переменных
$\sigma_{y_i}$ как $\boldsymbol{\sigma}_y$:
\[
  \boldsymbol{\sigma}_y = \{ \sigma_{y_1}, \dots, \sigma_{y_{\ell}} \}.
\]

Аналогично обозначим матрицу, составленную из ошибок измерений компонент
независимых переменных $\sigma_{x_{ij}}$ как $\Sigma_x$:
\[
  \Sigma_x = \| \sigma_{x_{ij}} \| | i \in \{ 1, \dots, \ell \}, j \in \{ 1, \dots, m \}.
\]

Требуется построить функционал ошибки $S(\bomega)$ вектора параметров
$\bomega$ модели $f$, учитывающий ошибки измерений $\sigma_{y_i}$ и
$\sigma_{x_{ij}}$:

\begin{equation}
  S(\bomega) = S(\bomega, \boldsymbol{\sigma}_y, \Sigma_x),
\end{equation}

и, кроме того, найти вектор параметров $\omega$, минимизирующий функционал
$S$:
\begin{equation}
  \hat{\bomega} = \mathop{\arg \min}\limits_{\bomega} S(\bomega)
\end{equation}

\section{Модифицированный функционал качества}

В качестве функционала ошибки $S$ предлагается использовать следующее выражение:

\begin{equation}
  S(\boldsymbol{\omega}) = \sum_{i = 1}^\ell \Big( \frac{y_i - f(\mathbf{x}_i, \boldsymbol{\omega})}{\sigma_{y_i} + \sum_{j = 1}^m \frac{\partial f}{\partial x_j}(\mathbf{x}_i, \boldsymbol{\omega}) \sigma_{x_{ij}}} \Big)^2.
\end{equation}

\FloatBarrier

\bibliographystyle{babunsrt-lf}
%\bibliographystyle{babunsrt}
%\bibliographystyle{unsrt}
\bibliography{bibliography}

\end{document}
